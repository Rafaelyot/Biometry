\section{Introdução}


\quad Este projeto baseia-se numa mudança protocolar relativamente ao primeiro projeto desta cadeira. Neste caso, o objetivo é o utilizador ter controlo ativo sobre os atributos de entidade usados durante os processos de autenticação, isto é, este tem plena noção que atributos de identidade estão a ser requisitados (por um  \textit{Service provider} (SP)) e o resultado da sua resolução (providenciado por um \textit{identity provider} (IDP)). Complementarmente à amostrem explicita dos atributos de entidades, o utilizador consegue tomar de decisões de consentimento (ou não) relativamente ao conjunto de atributos solicitados e resolvidos pelas entidades de autenticação. Outra caraterística vantajosa deste protocolo é a impossibilidade dos \textit{IDPs} conseguirem fazer \textit{track} de quais \textit{SPs} o utilizador interage.


